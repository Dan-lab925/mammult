\myprogram{{model\_layer\_growth.py}}
          {Layer growth with preferential activation model.} 
          {$<$layer\_N\_file$>$ $<$N$>$ $<$M0$>$ $<$A$>$ [RND]}

\mydescription{This is the model of layer growth with preferential
          node activation. In this model an entire new layer arrives
          at time $t$ and a number of nodes $N_t$ is activated ($N\_t$
          is equal to the number of nodes active on that layer in the
          reference multiplex). Then, each node $i$ of the new layer
          is activated with a probability:

          \begin{equation*}
          P_i(t) \propto A + B_i(t)
          \end{equation*}

          where $B_i(t)$ is the activity of node $i$ at time $t$
          (i.e., the number of layers in which node $i$ is active at
          time $t$) while $A>0$ is an intrinsic attractiveness.

          The file \textit{layer\_N\_file} reports on the n-th line
          the number of active nodes on the n-th layer.
          
          The parameter \textit{N} is the number of nodes in the
          multiplex, \textit{M0} is the number of layers in the
          initial network, \textit{A} is the value of
          node attractiveness.

          If the user specifies \texttt{RND} as the last parameter,
          the sequence of layers is }

\myreturn{The program prints on \texttt{stdout} a node-layer list of lines in the
 format:

 \hspace{0.5cm} \textit{node\_i layer\_i}
 
 where \textit{node\_i} is the ID of a node and \textit{layre\_i} is
 the ID of a layer. This list indicates which nodes are active in
 which layer. For instance, the line:

 \hspace{0.5cm} \textit{24 3}

 indicates that the node with ID \textit{24} is active on
 layer \textit{3}.
}

\myreference{\refcorrelations}
