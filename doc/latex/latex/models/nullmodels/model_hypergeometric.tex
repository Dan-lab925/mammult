\myprogram{{model\_hypergeometric.py}}
          {Hypergeometric node activity null model.} 
          {$<$layer\_N\_file$>$ $<$N$>$}

\mydescription{This is the hypergeometric model of node activation. In
          this model each layer has exactly the same number of active
          node of a reference multiplex network, but nodes on each
          layer are activated uniformly at random, thus destroying all
          inter-layer activity correlation patterns.

          The file \textit{layer\_N\_file} reports on the n-th line
          the number of active nodes on the n-th layer (starting from
          zero). The second parameter \textit{N} is the total number
          of active nodes in the multiplex.
 }


\myreturn{The program prints on \texttt{stdout} a node-layer list of lines in the
 format:

 \hspace{0.5cm} \textit{node\_i layer\_i}
 
 where \textit{node\_i} is the ID of a node and \textit{layre\_i} is
 the ID of a layer. This list indicates which nodes are active in
 which layer. For instance, the line:

 \hspace{0.5cm} \textit{24 3}

 indicates that the node with ID \textit{24} is active on
 layer \textit{3}.
}

\myreference{\refcorrelations}
