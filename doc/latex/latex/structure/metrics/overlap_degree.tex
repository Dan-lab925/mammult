\myprogram{{overlap\_degree.py}}
          {compute the total (overlapping) degree of all the nodes of
          a multiplex and the corresponding Z-score. }  {$<$layer1$>$ $<$layer2$>$ [$<$layer3$>$...]}

\mydescription{Compute and print on output the total degree $o_i$ of each
          node $i$ of a multiplex, defined as:

          \begin{equation*}
            o_{i} = \sum_{\alpha}\sum_{j}a_{ij}\lay{\alpha}
          \end{equation*}

   \noindent and the corresponding Z-score:

   \begin{equation*}
    z(o_i) = \frac{o_i - \avg{o}}{\sigma_o}
   \end{equation*}

  \noindent where $\avg{o}$ and $\sigma_o$ are, respectively, the mean
  and the standard deviation of the total degree computed over all the
  active nodes of the multiplex.


  Each input file contains the (undirected) edge list of a layer, and
  each line is in the format:
  
  \hspace{0.5cm}\textit{src\_ID} \textit{dest\_ID}
  
  where \textit{src\_ID} and \textit{dest\_ID} are the IDs of the two
  endpoints of an edge.}

\myreturn{The program prints on \texttt{stdout} a list of lines in the
  format:
  
  \hspace{0.5cm} \textit{ID\_n deg\_n z\_n}

  where \textit{ID\_n} is the ID of the node, \textit{deg\_n} is its
  total degree, and \textit{z\_n} is the corresponding Z-score.
  
  \noindent As usual, node IDs start from zero and proceed
  sequentially, without gaps, i.e., if a node ID is not present in any
  of the layer files given as input, the program considers it as being
  isolated on all the layers, and the node is omitted from the
  output.}

\myreference{\refmetrics}
