\myprogram{{cartography\_from\_layers.py}}
          {compute the total degree and the multiplex participation
          coefficient of all the nodes of a multiplex.}  {$<$layer1$>$
          $<$layer2$>$ [$<$layer3$>$...]}

\mydescription{Compute and print on output the total degree and the multiplex participation
          coefficient $P_i$ for each node $i$ of a multiplex. The
                       participation coefficient is defined as:

          \begin{equation*}
          P_i=\frac{M}{M-1}\left[1-\sum_{\alpha=1}^M\biggl(\frac{k_i^{[\alpha]}}{o_i}\biggr)^2\right] 
          \end{equation*}

  \noindent Note that $P_i$ takes values in $[0,1]$, where $P_i=0$
  if and only if node $i$ is active on exactly one of the layers,
  while $P_i=1$ if node $i$ has equal degree on all the $M$ layers.

  Each input file contains the (undirected) edge list of a layer, and
  each line is in the format:
  
  \hspace{0.5cm}\textit{src\_ID} \textit{dest\_ID}
  
  where \textit{src\_ID} and \textit{dest\_ID} are the IDs of the two
  endpoints of an edge.}

\myreturn{The program prints on \texttt{stdout} a list of lines in the
  format:
  
  \hspace{0.5cm} \textit{deg\_n P\_n col\_n}  

  where \textit{deg\_n} is the total degree of node $n$, \textit{P\_n}
  is the participation coefficient of node $n$ and \textit{col} is the
  integer representation of the activity bitstring of node $n$, which
  is a number between $0$ and $2^{M}-1$. The field \textit{col} might
  be useful for the visualisation of the multiplex cartography
  diagram, where it would be possible to associate different colors to
  nodes having different node activity patterns.
  
  \noindent As usual, node IDs start from zero and proceed
  sequentially, without gaps, i.e., if a node ID is not present in any
  of the layer files given as input, the program considers it as being
  isolated on all the layers, and is set to zero.
  }

\myreference{\refmetrics}
