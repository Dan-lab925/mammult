\myprogram{{layer\_activity\_vectors.py}}
          {compute the activity vectors of all the layers of a multiplex.}
          {$<$layer1$>$ [$<$layer2$>$ ...]}

\mydescription{Compute and print on output the activity vectors of the
  layers of a multiplex network, where the layers are given as input
  in the files \textit{layer1}, \textit{layer2}, etc.
  
  Each input file contains the (undirected) edge list of a layer, and
  each line is in the format:
  
  \hspace{0.5cm}\textit{src\_ID} \textit{dest\_ID}
  
  where \textit{src\_ID} and \textit{dest\_ID} are the IDs of the two
  endpoints of an edge.}

\myreturn{The program prints on \texttt{stdout} a list of lines, where
  the n-th line contains the activity vector of the n-th layer, i.e. a
  bit-string where each bit is set to ``1'' if the corresponding node
  is active on the n-th layer, and to ``0'' otherwise.
 
  \noindent As usual, node IDs start from zero and proceed
  sequentially, without gaps, i.e., if a node ID is not present in any
  of the layer files given as input, the program considers it as being
  isolated on all the layers.}

\myreference{\refcorrelations}
