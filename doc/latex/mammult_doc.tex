\documentclass[a4paper,11pt]{book}

\usepackage[british]{babel}
\usepackage[latin1]{inputenc}
\usepackage{hyperref}
\usepackage{amsmath, amssymb}

\newcommand{\lay}[1]{^{[#1]}}
\newcommand{\avg}[1]{\langle #1 \rangle}


\newcommand{\myprogram}[3]{\subsubsection{\texttt{#1}}
  \textbf{NAME}
    
  {\textbf{#1} - #2}
    
  \vspace{0.5cm}
  \noindent
  \textbf{SYNOPSYS}
  
  {\textbf{#1} { }\texttt{\textit{#3} } }
}

\newcommand{\mydescription}[1]{
  \vspace{0.5cm}
  \noindent
  \textbf{DESCRIPTION}

  {#1}
}

\newcommand{\myreturn}[1]{
  \vspace{0.5cm}
  \noindent
  \textbf{OUTPUT}

  {#1}
}


\newcommand{\myreference}[1]{
  \vspace{0.5cm}
  \noindent
  \textbf{REFERENCE}

  {#1}
}

%%
%% REFERENCES
%%

\newcommand{\refgrowth}{V. Nicosia, G. Bianconi, V. Latora,
  M. Barthelemy, ``Growing multiplex networks'',
  \textit{Phys. Rev. Lett.} {\bf 111}, 058701 (2013).
  
  Link to paper: \url{http://prl.aps.org/abstract/PRL/v111/i5/e058701}
}
\newcommand{\refnonlinear}{V. Nicosia, G. Bianconi, V. Latora,
  M. Barthelemy, ``Non-linear growth and condensation in multiplex
  networks'', \textit{Phys. Rev. E} {\bf 90}, 042807 (2014).
  
  Link to paper: \url{http://journals.aps.org/pre/abstract/10.1103/PhysRevE.90.042807}
}
\newcommand{\refmetrics}{F. Battiston, V. Nicosia, V. Latora,
  ``Structural measures for multiplex networks'',
  \textit{Phys. Rev. E} {\bf 89}, 032804 (2014).
  
  Link to paper: \url{http://journals.aps.org/pre/abstract/10.1103/PhysRevE.89.032804}
}
\newcommand{\refcorrelations}{V. Nicosia, V. Latora, ``Measuring and
  modeling correlations in multiplex networks'', \textit{Phys. Rev. E}
  {\bf 92}, 032805 (2015).
  
  Link to paper: \url{http://journals.aps.org/pre/abstract/10.1103/PhysRevE.92.032805}}

\newcommand{\refreducibility}{M. De Domenico, V. Nicosia, A. Arenas,
  V. Latora, \textit{``Structural reducibility of multilayer
    networks''}, Nat. Commun. {\bf 6}, 6864 (2015).
  
  Link to paper: \url{http://www.nature.com/ncomms/2015/150423/ncomms7864/full/ncomms7864.html}
}
\newcommand{\refvisibility}{L. Lacasa, V. Nicosia, V. Latora,
  \textit{``Network structure of multivariate time series''}, accepted
  for publication in Scientific Reports, arxiv:1408.0925 (2015).
  
  Link to paper: \url{http://arxiv.org/abs/1408.0925}
}

\newcommand{\refbiased}{F. Battiston, V. Nicosia, V. Latora,
  \textit{``Biased random walks on multiplex networks''}, arxiv:1505.01378 (2015).
  
  Link to paper: \url{http://arxiv.org/abs/1505.01378}
}
\newcommand{\refising}{F. Battiston, A. Cairoli, V. Nicosia, A. Baule, V. Latora,
  \textit{``Interplay between consensus and coherence in a model of interacting opinions''}, accepted
  for publication in Physica D, arxiv:1506.04544 (2015).
  
  Link to paper: \url{http://arxiv.org/abs/1506.04544}
}
\newcommand{\refcommunity}{F. Battiston, J. Iacovacci, V. Nicosia, G. Bianconi, V. Latora,
  \textit{``Emergence of multiplex communities in collaboration networks''}, arxiv:1506.01280 (2015).
  
  Link to paper: \url{http://arxiv.org/abs/1506.01280}
}

\newcommand{\refaxelrod}{F. Battiston, V. Nicosia, V. Latora, M. San Miguel,
  \textit{``Layered social influence promotes multiculturality''}, in preparation (2015).
  
  %Link to paper: \url{http://arxiv.org/abs/inpreparation}
}
\newcommand{\refmotifs}{F. Battiston, M. Chavez, V. Nicosia, V. Latora,
  \textit{``Multilayer motifs in brain networks''}, in preparation (2015).
  
  %Link to paper: \url{http://arxiv.org/abs/inpreparation}
}

\title{MAMMULT: Metrics And Models for MULTilayer networks}


\begin{document}

\maketitle

\chapter{Structural descriptors}

\section{Basic node, edge, and layer properties}

\subsection{Node and layer activity}

This section includes programs related to the computation of node and
layer activity, activity vectors, pairwise multiplexity, pairwise
normalised Hamming distance, node degree vectors.

%%%
%%% node_activity
%%%
\myprogram{{node\_activity.py}}
          {compute the activity of the nodes of a multiplex, i.e. the
            number of layers where each node is not isolated.}
          {$<$layer1$>$ [$<$layer2$>$ ...]}

\mydescription{Compute and print on output the activity of the nodes
  of a multiplex network, whose layers are given as input in the files
  \textit{layer1}, \textit{layer2}, etc.
  
  Each file contains the (undirected) edge list of a layer, and each
  line is in the format:
  
  \hspace{0.5cm}\textit{src\_ID} \textit{dest\_ID}
  
  where \textit{src\_ID} and \textit{dest\_ID} are the IDs of the two
  endpoints of an edge.}

\myreturn{A list of lines, where the n-th line is the value of activity
  of the n-th node, starting from \textbf{0}.}

\myreference{\refcorrelations}


%%%
%%% Layer activity
%%%

\myprogram{{layer\_activity.py}}
          {compute the activity of the layers of a multiplex, i.e. the
          number of active nodes on each layer.}
          {$<$layer1$>$ [$<$layer2$>$ ...]}

\mydescription{Compute and print on output the activity of the layers
  of a multiplex network, where the layers are given as input in the
  files \textit{layer1}, \textit{layer2}, etc.
  
  Each file contains the (undirected) edge list of a layer, and each
  line is in the format:
  
  \hspace{0.5cm}\textit{src\_ID} \textit{dest\_ID}
  
  where \textit{src\_ID} and \textit{dest\_ID} are the IDs of the two
  endpoints of an edge.}

\myreturn{A listof lines, where the n-th line is the value of activity
  of the n-th layer, starting from \textbf{0}.}

\myreference{\refcorrelations}

\myprogram{{node\_activity\_vectors.py}}
          {compute the activity vectors of all the nodes of a multiplex.}
          {$<$layer1$>$ [$<$layer2$>$ ...]}

\mydescription{Compute and print on output the activity vectors of the
  nodes of a multiplex network, whose layers are given as input in the
  files \textit{layer1}, \textit{layer2}, etc.
  
  Each input file contains the (undirected) edge list of a layer, and
  each line is in the format:
  
  \hspace{0.5cm}\textit{src\_ID} \textit{dest\_ID}
  
  where \textit{src\_ID} and \textit{dest\_ID} are the IDs of the two
  endpoints of an edge.}

\myreturn{The program prints on \texttt{stdout} a list of lines, where
  the n-th line contains the activity vector of the n-th node, i.e. a
  bit-string where each bit is set to ``1'' if the node is active on
  the corresponding layer, and to ``0'' otherwise.
 
  \noindent As usual, node IDs start from zero and proceed
  sequentially, without gaps, i.e., if a node ID is not present in any
  of the layer files given as input, the program considers it as being
  isolated on all the layers, and will print on output a bit-string of
  zeros.}

\myreference{\refcorrelations}

\myprogram{{layer\_activity\_vectors.py}}
          {compute the activity vectors of all the layers of a multiplex.}
          {$<$layer1$>$ [$<$layer2$>$ ...]}

\mydescription{Compute and print on output the activity vectors of the
  layers of a multiplex network, where the layers are given as input
  in the files \textit{layer1}, \textit{layer2}, etc.
  
  Each input file contains the (undirected) edge list of a layer, and
  each line is in the format:
  
  \hspace{0.5cm}\textit{src\_ID} \textit{dest\_ID}
  
  where \textit{src\_ID} and \textit{dest\_ID} are the IDs of the two
  endpoints of an edge.}

\myreturn{The program prints on \texttt{stdout} a list of lines, where
  the n-th line contains the activity vector of the n-th layer, i.e. a
  bit-string where each bit is set to ``1'' if the corresponding node
  is active on the n-th layer, and to ``0'' otherwise.
 
  \noindent As usual, node IDs start from zero and proceed
  sequentially, without gaps, i.e., if a node ID is not present in any
  of the layer files given as input, the program considers it as being
  isolated on all the layers.}

\myreference{\refcorrelations}

\myprogram{{multiplexity.py}}
          {compute the pairwise multiplexity between all the pairs of
          layers of a multiplex.}
          {$<$layer1$>$ $<$layer2$>$ [$<$layer3$>$...]}

\mydescription{Compute and print on output the pairwise multiplexity 
 $Q_{\alpha, \beta}$ (i.e., the fraction of nodes active on both
          layers) between all pairs of layers. The layers are given as
          input in the files \textit{layer1}, \textit{layer2}, etc.
  
  Each input file contains the (undirected) edge list of a layer, and
  each line is in the format:
  
  \hspace{0.5cm}\textit{src\_ID} \textit{dest\_ID}
  
  where \textit{src\_ID} and \textit{dest\_ID} are the IDs of the two
  endpoints of an edge.}

\myreturn{The program prints on \texttt{stdout} a list of lines, in
  the format:

  \hspace{0.5cm} \textit{layer1 layer2 mult}

  \noindent where \textit{layer1} and \textit{layer2} are the IDs of
  the layers, and \textit{mult} is the value of the multiplexity
  $Q_{layer1, layer2}$. Layers IDs start from zero, are are associated
  to the layers in the same order in which the layer files are
  provided on the command line.}

\myreference{\refcorrelations}

\myprogram{{hamming\_dist.py}}
          {compute the normalised Hamming distance between all the pairs of
          layers of a multiplex.}
          {$<$layer1$>$ $<$layer2$>$ [$<$layer3$>$...]}

\mydescription{Compute and print on output the normalised Hamming distance 
 $H_{\alpha, \beta}$ (i.e., the fraction of nodes which are active on
          either of the layers, but not on both) between all pairs of
          layers. The layers are given as input in the
          files \textit{layer1}, \textit{layer2}, etc.
  
  Each input file contains the (undirected) edge list of a layer, and
  each line is in the format:
  
  \hspace{0.5cm}\textit{src\_ID} \textit{dest\_ID}
  
  where \textit{src\_ID} and \textit{dest\_ID} are the IDs of the two
  endpoints of an edge.}

\myreturn{The program prints on \texttt{stdout} a list of lines, in
  the format:

  \hspace{0.5cm} \textit{layer1 layer2 hamm}

  \noindent where \textit{layer1} and \textit{layer2} are the IDs of
  the layers, and \textit{hamm} is the value of the normalised Haming
  distance $H_{layer1, layer2}$. Layers IDs start from zero, are are
  associated to the layers in the same order in which the layer files
  are provided on the command line.}

\myreference{\refcorrelations}

\myprogram{{node\_degree\_vectors.py}}
          {compute the degree vectors of all the nodes of a multiplex network}
          {$<$layer1$>$ [$<$layer2$>$ ...]}

\mydescription{Compute and print on output the degree vectors of all
          the nodes of a multiplex network, whose layers are given as
  input in the files \textit{layer1}, \textit{layer2}, etc.
  
  Each file contains the (undirected) edge list of a layer, and each
  line is in the format:
  
  \hspace{0.5cm}\textit{src\_ID} \textit{dest\_ID}
  
  where \textit{src\_ID} and \textit{dest\_ID} are the IDs of the two
  endpoints of an edge.}

\myreturn{A list of lines, where the n-th line is the 
  vector of degrees of the n-th node, in the format:
  
  \hspace{0.5cm}\textit{noden\_deg\_lay1 noden\_deg\_lay2 ... noden\_deg\_layM}

  \noindent As usual, node IDs start from zero and proceed
  sequentially, without gaps, i.e., if a node ID is not present in any
  of the layer files given as input, the program considers it as being
  isolated on all the layers.

}

\myreference{\refgrowth\\ \\ \indent \refmetrics}


\myprogram{{degs\_to\_binary.py}}
          {compute the activity vectors of all the nodes of a multiplex.}
          {$<$degree\_vectors$>$}

\mydescription{Take a file which contains, on the n-th line, the degrees at each
 layer of the n-th node, (e.g., the result of the
 script \texttt{node\_degree\_vectors.py}), in the format:

 \hspace{0.5cm}\textit{noden\_deg\_lay1 noden\_deg\_lay2 ... noden\_deg\_layM}

 \noindent and compute the corresponding node activity bit-strings,
 where a "1" signals the presence of the node on that layer, while a
 zero indicates its absence.
}

\myreturn{The program returns on \texttt{stdout} a list of lines,
  where the n-th line is the activity bit-string of the n-th
  node. Additionally, the program prints on \texttt{stderr} the
  distribution of all activity bit-strings, in the format:

  \hspace{0.5cm}\textit{Bn Bit-string count}

  \noindent Where \textit{B} is the number of ones in the activity
  bit-string (i.e., the node-activity associated to that activity
  bit-string), \textit{Bit-string} is the activity bit-string
  and \textit{count} is the number of times that particular activity
  bit-string appears in the multiplex.}
  


\myreference{\refcorrelations}


\myprogram{{degs\_to\_activity\_overlap.py}}
          {compute the activity and the total (overlapping) degree of
          all the nodes of a multiplex.}
          {$<$degree\_vectors$>$}

\mydescription{Take a file which contains, on the n-th line, the degrees at each
 layer of the n-th node, (e.g., the result of the
 script \texttt{node\_degree\_vectors.py}), in the format:

 \hspace{0.5cm}\textit{noden\_deg\_lay1 noden\_deg\_lay2 ... noden\_deg\_layM}

 \noindent and compute the activity (i.e., the number of layers in
 which a node is not isolated) and the total (overlapping) degree of
 each node.}

\myreturn{The program prints on \texttt{stdout} a list of lines, where
 the n-th line contains the activity and the total degree of the n-th
 nodem in the format:

 \hspace{0.5cm}\textit{noden\_activity noden\_tot\_deg}

 \noindent As usual, the program assumes that node IDs start from zero
  and proceed sequentially, without gaps, i.e., if a node ID is not
  present in any of the layer files given as input, the program
  considers it as being isolated on all the layers.
 }

\myreference{\refcorrelations}



\subsection{Layer aggregation}

This section includes programs to obtain various single-layer
aggregated graphs associated to a multiplex network. 

\myprogram{{aggregate\_layers\_w.py}}
          {compute the (weighted) aggregated graph associated to a
          multiplex.}  {$<$layer1$>$ $<$layer2$>$ [$<$layer3$>$...]}

\mydescription{Compute and print on output the edge list of the  
               weighted aggregated graph associated to the multiplex
               network given on input. An edge is present in the
               aggregated graph if it exists in at least one of the M
               layers of the multiplex.

  Each input file contains the (undirected) edge list of a layer, and
  each line is in the format:
  
  \hspace{0.5cm}\textit{src\_ID} \textit{dest\_ID}
  
  where \textit{src\_ID} and \textit{dest\_ID} are the IDs of the two
  endpoints of an edge.}

\myreturn{The program prints on \texttt{stdout} the edge list of the
  aggregated graph associated to the multiplex network. The edge list
  is a list of lines in the format:
   

  \hspace{0.5cm} \textit{ID1 ID2 weight}

  \noindent where \textit{ID1} and \textit{ID2} are the IDs of the two
  nodes and \textit{weight} is the number of layers in which an edge
  between \textit{ID1} and \textit{ID2} exists.}

\myreference{\refcorrelations}

\myprogram{{intersect\_layers.py}}
          {compute the intersection graph associated to a
          multiplex.}  {$<$layer1$>$ $<$layer2$>$ [$<$layer3$>$...]}

\mydescription{Compute and print on output the edge list of the  
               intersection graph associated to the multiplex network
               given on input, where an edge exists only if it is
               present on \textbf{all} the layers of the multiplex.

  Each input file contains the (undirected) edge list of a layer, and
  each line is in the format:
  
  \hspace{0.5cm}\textit{src\_ID} \textit{dest\_ID}
  
  where \textit{src\_ID} and \textit{dest\_ID} are the IDs of the two
  endpoints of an edge.}

\myreturn{The program prints on \texttt{stdout} the edge list of the
  intersection graph associated to the multiplex network. The edge
  list is a list of lines in the format:
   

  \hspace{0.5cm} \textit{ID1 ID2}

  \noindent where \textit{ID1} and \textit{ID2} are the IDs of the two
  nodes.}

\myreference{\refcorrelations}


\subsection{Node degree, participation coefficient, cartography}

This section includes programs to compute the total degree and
participation coefficient of each node, and to draw the cartography
diagram of a multiplex.

\myprogram{{overlap\_degree.py}}
          {compute the total (overlapping) degree of all the nodes of
          a multiplex and the corresponding Z-score. }  {$<$layer1$>$ $<$layer2$>$ [$<$layer3$>$...]}

\mydescription{Compute and print on output the total degree $o_i$ of each
          node $i$ of a multiplex, defined as:

          \begin{equation*}
            o_{i} = \sum_{\alpha}\sum_{j}a_{ij}\lay{\alpha}
          \end{equation*}

   \noindent and the corresponding Z-score:

   \begin{equation*}
    z(o_i) = \frac{o_i - \avg{o}}{\sigma_o}
   \end{equation*}

  \noindent where $\avg{o}$ and $\sigma_o$ are, respectively, the mean
  and the standard deviation of the total degree computed over all the
  active nodes of the multiplex.


  Each input file contains the (undirected) edge list of a layer, and
  each line is in the format:
  
  \hspace{0.5cm}\textit{src\_ID} \textit{dest\_ID}
  
  where \textit{src\_ID} and \textit{dest\_ID} are the IDs of the two
  endpoints of an edge.}

\myreturn{The program prints on \texttt{stdout} a list of lines in the
  format:
  
  \hspace{0.5cm} \textit{ID\_n deg\_n z\_n}

  where \textit{ID\_n} is the ID of the node, \textit{deg\_n} is its
  total degree, and \textit{z\_n} is the corresponding Z-score.
  
  \noindent As usual, node IDs start from zero and proceed
  sequentially, without gaps, i.e., if a node ID is not present in any
  of the layer files given as input, the program considers it as being
  isolated on all the layers, and the node is omitted from the
  output.}

\myreference{\refmetrics}

\myprogram{{cartography\_from\_layers.py}}
          {compute the total degree and the multiplex participation
          coefficient of all the nodes of a multiplex.}  {$<$layer1$>$
          $<$layer2$>$ [$<$layer3$>$...]}

\mydescription{Compute and print on output the total degree and the multiplex participation
          coefficient $P_i$ for each node $i$ of a multiplex. The
                       participation coefficient is defined as:

          \begin{equation*}
          P_i=\frac{M}{M-1}\left[1-\sum_{\alpha=1}^M\biggl(\frac{k_i^{[\alpha]}}{o_i}\biggr)^2\right] 
          \end{equation*}

  \noindent Note that $P_i$ takes values in $[0,1]$, where $P_i=0$
  if and only if node $i$ is active on exactly one of the layers,
  while $P_i=1$ if node $i$ has equal degree on all the $M$ layers.

  Each input file contains the (undirected) edge list of a layer, and
  each line is in the format:
  
  \hspace{0.5cm}\textit{src\_ID} \textit{dest\_ID}
  
  where \textit{src\_ID} and \textit{dest\_ID} are the IDs of the two
  endpoints of an edge.}

\myreturn{The program prints on \texttt{stdout} a list of lines in the
  format:
  
  \hspace{0.5cm} \textit{deg\_n P\_n col\_n}  

  where \textit{deg\_n} is the total degree of node $n$, \textit{P\_n}
  is the participation coefficient of node $n$ and \textit{col} is the
  integer representation of the activity bitstring of node $n$, which
  is a number between $0$ and $2^{M}-1$. The field \textit{col} might
  be useful for the visualisation of the multiplex cartography
  diagram, where it would be possible to associate different colors to
  nodes having different node activity patterns.
  
  \noindent As usual, node IDs start from zero and proceed
  sequentially, without gaps, i.e., if a node ID is not present in any
  of the layer files given as input, the program considers it as being
  isolated on all the layers, and is set to zero.
  }

\myreference{\refmetrics}

\myprogram{{cartography\_from\_deg\_vectors.py}}
          {create a multiplex cartography diagram.}  
          {$<$node\_deg\_vectors$>$}

\mydescription{Compute and print on output the total degree and the
          multiplex participation coefficient of all the nodes of a
          multiplex network whose list of node degree vectors is
          provided as input. The input file is in the format:

  \hspace{0.5cm} \textit{IDn\_deg1 IDn\_deg\_2 ... IDn\_degM}

  \noindent where \textit{IDn\_degX} is the degree of node $n$ at
  layer $X$. The input file can be generated using the
  script \texttt{node\_degree\_vectors.py}.}


\myreturn{The program prints on \texttt{stdout} a list of lines in the
  format:
  
  \hspace{0.5cm} \textit{tot\_deg part\_coeff}  

  \noindent where \textit{tot\_deg} is the total degree of the node
  and \textit{part\_coeff} is the corresponding participation
  coefficient. 
  
  \noindent As usual, node IDs start from zero and proceed
  sequentially, without gaps, so if one of the lines in the input
  files contains just zeros, the program considers the corresponding
  node as being isolated on all the layers, and both its total degree
  and multiplex participation coefficient are set equal to zero. }

\myreference{\refmetrics}

\myprogram{{cartography\_from\_columns.py}}
          {compute total and participation coefficient of generic
          structural descriptors of the nodes of a multiplex.}
          {$<$filein$>$ $<$col1$>$ $<$col2$>$ [$<$col3$>$...]}

\mydescription{Compute and print on output the sum and the
          corresponding participation coefficient of a generic
          structural descriptor of the nodes of a multiplex.

   \noindent The input file is a generic collection of
   single-space-separated columns, where each line corresponds to a
   node. The user must specify the IDs of the columns which contain
   the node structural descriptors to be used in the cartography
   diagram. Columns IDs start from ZERO. For example:

   \textbf{python cartography\_from\_layers.py filein.txt 0
   2 4 6 8}

   \noindent will create a cartography diagram assuming that the
   multiplex network has five layers, and that the node structural
   descriptors at each layers are contained in the first (0), third
   (2), fifth (4), seventh (6) and nineth (8) columns of each row.}


\myreturn{The program prints on \texttt{stdout} a list of lines in the
  format:
  
  \hspace{0.5cm} \textit{tot\_n P\_n}

  where \textit{tot\_n} is the sum over the layers of the considered
  structural descriptor for node $n$, and \textit{P\_n} is the
  associated participation coefficient 
  }

\myreference{\refcorrelations}


\subsection{Edge overlap, reinforcement}
This section includes programs to compute the egde overlap and to
evaulate the edge reinforcement effect. 

\myprogram{{edge\_overlap.py}}
          {compute the edge overlap of all the edges of the
          multiplex.}  
          {$<$layer1$>$ [$<$layer2$>$...]}

\mydescription{Compute and print on output the edge overlap $o_{ij}$ of each
          edge of the multiplex. Given a pair of nodes $(i,j)$ that
          are directly connected on at least one of the $M$ layers,
          the edge overlap $o_{ij}$ is defined as:

          \begin{equation*}
          o_{ij} = \sum_{\alpha}a_{ij}\lay{\alpha}
          \end{equation*}
          
  \noindent i.e., the number of layers on which the edge $(i,j)$
  exists.
  
  
  Each input file contains the (undirected) edge list of a layer, and
  each line is in the format:
  
  \hspace{0.5cm}\textit{src\_ID} \textit{dest\_ID}
  
  where \textit{src\_ID} and \textit{dest\_ID} are the IDs of the two
  endpoints of an edge.}

\myreturn{The program prints on \texttt{stdout} a list of lines in the
  format:

  \hspace{0.5cm} \textit{ID\_1 ID\_2 overlap}

  \noindent where \textit{ID\_1} and \textit{ID\_2} are the IDs of the
  end-points of the edge, and \textit{overlap} is the number of layers
  in which the edge exists.}

\myreference{\refmetrics}

\myprogram{{avg\_edge\_overlap.py}}
          {compute the average edge overlap of a multiplex.}  
          {$<$layer1$>$ [$<$layer2$>$...]}

\mydescription{Compute and print on output the average edge overlap 

          \begin{equation*} \omega^{*}
          = \frac{\sum_{i}\sum_{j>i}\sum_{\alpha}a_{ij}\lay{\alpha}}{ \sum_{i}\sum_{j>i}(1
          - \delta_{0,\sum_{\alpha}a_{ij}\lay{\alpha}})} \end{equation*}
          
  \noindent i.e., the expected \textit{number} of layers on which an
   edge of the multiplex exists, and the corresponding normalised
   quantity:
  
  \begin{equation*}
      \omega = \frac{\sum_{i}\sum_{j>i}\sum_{\alpha}a_{ij}\lay{\alpha}}{M \sum_{i}\sum_{j>i}(1
      - \delta_{0,\sum_{\alpha}a_{ij}\lay{\alpha}})}
  \end{equation*}
  
  \noindent that is the expected \textit{fraction} of layers on which
  an edge of the multiplex is present.
  
  Each input file contains the (undirected) edge list of a layer, and
  each line is in the format:
  
  \hspace{0.5cm}\textit{src\_ID} \textit{dest\_ID}
  
  where \textit{src\_ID} and \textit{dest\_ID} are the IDs of the two
  endpoints of an edge.}

\myreturn{The program prints on \texttt{stdout} a single line, in the
  format:
  
  \hspace{0.5cm} \textit{omega\_star omega}

  \noindent where \textit{omega\_star} and \textit{omega} are,
  respectively, the expected number and fraction of layers in which an
  edge is present.}

\myreference{\refmetrics

  \vspace{0.5cm}\refvisibility}

%%%
%%% Layer activity
%%%

\myprogram{{reinforcement.py}}
          {compute the probability to have a link between two nodes in layer $1$ given their weight in layer $2$.}
          {$<$layer1$>$ $<$layer2$>$ $<N_{bins}>$ $<min_{value}>$ $<max_{value}>$}

\mydescription{Compute and print on output the probability to have a link between two nodes in layer $1$ given their weight in layer $2$. 
As input are given the files \textit{layer1}, \textit{layer2}, the number of bins for the link weights of the second layer, the minimum and the maximum values of the binning.
  
  The first file contains the binary edge list of layer $1$, the second file contains the weighted edge list of layer $2$. each
  line is in the format:
  
  \hspace{0.5cm}\textit{bin\_min} \textit{bin\_max} \textit{freq}
  
  where \textit{bin\_min} and \textit{bin\_max} are the minimum and maximum values of the link weights of layer $2$ in that binning, and  \textit{freq} is the probability to have a link on layer $1$ given such weight in layer $2$.}

\myreturn{A list of lines, where the n-th line is the minimum and maximum values of the weight of the links in layer $2$ in the n-th bin, and the frequency to have a link on layer $2$ given that weight.}

\myreference{\refmetrics}


\section{Inter-layer degree correlations}

\subsection{Node ranking}

This section includes various utilities to compute and compare node
rankings induced by any generic structural node property, including
degree at different layers.

\myprogram{{rank\_nodes.py}}
          {rank the nodes of a layer according to a given structural
          descriptor.}  {$<$prop\_file$>$}

\mydescription{Get a file as input, whose n-th line corresponds to the value of a
 certain property of the n-th node, and rank the nodes according to
 that property, taking into account ranking ties properly.
 
 For example, if \textit{propfile} contains the degrees of the nodes
 at a certain layer of the multiplex, the computes the ranking induced
 by degrees, where the node with the highest degree will be assigned a
 rank equal to \textbf{1} (one).
}

\myreturn{The program prints on \texttt{stdout} a list of lines, where
 the n-th line contains the rank of the n-th node corresponding to the
 values of the structural descriptor provided in the input file.}

\myreference{\refcorrelations}

\myprogram{{rank\_nodes\_thresh.py}}
          {rank the nodes of a layer whose value of a given structural
          descriptor is above a threshold.}  {$<$prop\_file$>$ $<$thresh$>$}

\mydescription{Get a file as input, whose n-th line corresponds to the value of a
 certain property of the n-th node, and rank the nodes according to
 that property, taking into account ranking ties properly. The rank of
 all the nodes whose value of the structural descriptor is smaller
 than the threshold \textit{thresh} specified as second parameter is
 set to \textbf{0} (ZERO). }

\myreturn{The program prints on \texttt{stdout} a list of lines, where
 the n-th line contains the rank of the n-th node corresponding to the
 values of the structural descriptor provided in the input file, or
 zero if such desxriptor is below the specified
 threshold \textit{thresh}.}

\myreference{\refcorrelations}

\myprogram{{rank\_occurrence.py}}
          {compute the intersection of two rankings.}  
          {$<$rank1$>$ $<$rank2$>$ $<$increment$>$}

\mydescription{Get two rankings \textit{rank1} and \textit{rank2} 
                   and compute the size of
the \textit{k}-intersection, i.e. the number of elements which are
present in the first k positions of both rankings, as a function
of \textit{k}.  The parameter \textit{increment} determines the
distance between two subsequent values of \textit{k}.

Each input file is a list of node IDs, one per line, where the first
line contains the ID of the highest ranked node.
}

\myreturn{The program prints on \texttt{stdout} a list of lines in the
format:
  
  \hspace{0.5cm} \textit{k num\_k}
  
  where \textit{num\_k} is the number of nodes which are present in
  the first \textit{k} positions of both rankings.}

\myreference{\refcorrelations}


\subsection{Interlayer degree correlation coefficients}

This section includes programs for the computation of various
inter-layer degree correlation coefficients.

\myprogram{{compute\_pearson.py}}
          {compute the Pearson's linear correlation coefficient
          between two node properties.}  
          {$<$file1$>$ $<$file2$>$}

\mydescription{Compute the Pearson's linear correlation coefficient
          between two sets of (either integer- or real-valued) node
          properties provided in the input files \textit{file1}
          and \textit{file2}. Each input file contains a list of
          lines, where the n-th line contains the value of a node
          property for the n-th node.  For instance, \textit{file1}
          and \textit{file2} might contain the degrees of nodes at two
          distinct layers of a multiplex. However, the program is
          pretty general and can be used to compute the Pearson's
          correlation coeffcient between any pairs of node properties.
          }

\myreturn{The program prints on \texttt{stdout} the value of the
          Pearson's linear correlation coefficient between the two
          sets of node properties.
}

\myreference{\refcorrelations

  \refgrowth

  \refnonlinear
}

\myprogram{{compute\_rho.py}}
          {compute the Spearman's rank correlation coefficient $\rho$
          between two rankings.}  {$<$file1$>$ $<$file2$>$}

\mydescription{Compute the Spearman's rank correlation coefficient
           $\rho$ between two rankings provided in the input
          files \textit{file1} and \textit{file2}. Each input file
          contains a list of lines, where the n-th line contains the
          value of rank of the n-th node. For instance, \textit{file1}
          and \textit{file2} might contain the ranks of nodes induced
          by the degree sequences of two distinct layers of a
          multiplex. 

          However, the program is pretty general and can be used to
          compute the Spearman's rank correlation coefficient between
          any generic pair of rankings.

          N.B.: A C implementation of this program, with the same
          interface is also available in the executable
          file \texttt{compute\_rho}.}


\myreturn{The program prints on \texttt{stdout} the value of the
          Spearman's rank correlation coefficient $\rho$ between the
          two rankings provided as input. }

\myreference{\refcorrelations

  \refgrowth

  \refnonlinear
  }

\myprogram{{compute\_tau.py}}
          {compute the Kendall's rank correlation coefficient $\tau_b$
          between two rankings.}  {$<$file1$>$ $<$file2$>$}

\mydescription{Compute the Kendall's rank correlation coefficient
           $\tau_b$ between two rankings provided in the input
          files \textit{file1} and \textit{file2}. Each input file
          contains a list of lines, where the n-th line contains the
          value of rank of the n-th node. For instance, \textit{file1}
          and \textit{file2} might contain the ranks of nodes induced
          by the degree sequences of two distinct layers of a
          multiplex.

          However, the program is pretty general and can be used to
          compute the Kendall's rank correlation coefficient between
          any generic pair of rankings.

          N.B.: This implementation takes properly into account rank
          ties.}


\myreturn{The program prints on \texttt{stdout} the value of the
          Kendall's rank correlation coefficient $\tau_b$ between the
          two rankings provided as input. }

\myreference{\refcorrelations

  \refgrowth

  \refnonlinear
}



\subsection{Interlayer degree correlation functions}

This section includes programs to compute intra-layer and inter-layer
degree correlation functions, and to fit those functions with a
power-law.


M\myprogram{{dump\_k\_q}}
          {compute the degree sequences of two layers of a multiplex.}
          {$<$layer1$>$ $<$layer2$>$ $<$pairing$>$}

\mydescription{Compute and dump on \texttt{stdout} the degree
          sequences of two layers of a multiplex. The input
          files \textit{layer1} and \textit{layer2} contain the
          (undirected) edge lists of the two layers, and each line is
          in the format:
  
  \hspace{0.5cm}\textit{src\_ID} \textit{dest\_ID}
  
  where \textit{src\_ID} and \textit{dest\_ID} are the IDs of the two
  endpoints of an edge.

  The third file \textit{pairing} is a list of lines in the format:

  \hspace{0.5cm} \textit{IDL1 IDL2} 

  where \textit{IDL1} is the ID of a node on layer $1$
  and \textit{IDL2} is the ID of the same node on layer $2$. For
  instance, the line:

  \hspace{0.5cm} \textit{5 27}

  indicates that node $5$ on layer $1$ has ID $27$ on layer $2$. }


\myreturn{The program prints on \texttt{stdout} the degree of each
node on the two layers, in the format:

\hspace{0.5cm} \textit{ki qi}

where \textit{ki} is the degree of node \textit{i} on layer $1$
and \textit{qi} is the degree of node \textit{i} on layer $2$.}

\myreference{\refcorrelations

  \refgrowth

  \refnonlinear
  }

\myprogram{{knn\_q\_from\_layers.py}}
          {compute intra-layer and inter-layer degree-degree
          correlation coefficients.}  {$<$layer1$>$ $<$layer2$>$}

\mydescription{Compute the intra-layer and the inter-layer degree
          correlation functions for two layers given as input. The
          intra-layer degree correlation function quantifies the
          presence of degree-degree correlations in a single layer
          network, and is defined as:

          \begin{equation*}
            \avg{k_{nn}(k)} = \frac{1}{k N_k}\sum_{k'}k'P(k'|k)
          \end{equation*}
          
          where $P(k'|k)$ is the probability that a neighbour of a
          node with degree $k$ has degree $k'$, and $N_k$ is the
          number of nodes with degree $k$. The quantity
          $\avg{k_{nn}(k)}$ is the average degree of the neighbours of
          nodes having degree equal to $k$.

          If we consider two layers of a multiplex, and we denote by
          $k$ the degree of a node on the first layer and by $q$ the
          degree of the same node on the second layers, the
          inter-layer degree correlation function is defined as

          \begin{equation*}
            \overline{k}(q) = \sum_{k'} k' P(k'|q)
          \end{equation*}
           
           where $P(k'|q)$ is the probability that a node with degree
           $q$ on the second layer has degree equal to $k'$ on the
           first layer, and $N_q$ is the number of nodes with degree
           $q$ on the second layer. The quantity $\overline{k}(q)$ is
           the expected degree at layer $1$ of node that have degree
           equal to $q$ on layer $2$. The dual quantity:

          \begin{equation*}
            \overline{q}(k) = \sum_{q'} q' P(q'|k)
          \end{equation*}
           
           is the average degree on layer $2$ of nodes having degree
           $k$ on layer $1$.
}


\myreturn{The program creates two output files, respectively called

\hspace{0.5cm} \textit{file1\_file2\_k1}

and

\hspace{0.5cm} \textit{file1\_file2\_k2}

The first file contains a list of lines in the format:

\hspace{0.5cm} \textit{k $\avg{k_{nn}(k)}$ $\sigma_k$
$\overline{q}(k)$ $\sigma_{\overline{q}}$}

where $k$ is the degree at first layer, $\avg{k_{nn}(k)}$ is the
average degree of the neighbours at layer $1$ of nodes having degree
$k$ at layer $1$, $\sigma_k$ is the standard deviation associated to
$\avg{k_{nn}(k)}$, $\overline{q}(k)$ is the average degree at layer
$2$ of nodes having degree equal to $k$ at layer $1$, and
$\sigma_{\overline{q}}$ is the standard deviation associated to
$\overline{q}(k)$. 

The second file contains a similar list of lines, in the format:

\hspace{0.5cm} \textit{q $\avg{q_{nn}(q)}$ $\sigma_q$
$\overline{k}(q)$ $\sigma_{\overline{k}}$}

with obvious meaning.
}

\myreference{\refcorrelations

  \refgrowth

  \refnonlinear
  }

%% \input{./latex/structure/correlations/knn_q_from_layers_log.tex}
\myprogram{{knn\_q\_from\_degrees.py}}
          {compute the inter-layer degree-degree correlation function.}
          {$<$filein$>$}

\mydescription{Compute the  inter-layer degree
          correlation functions for two layers of a multiplex, using
          the degrees of the nodes specified in the input file. The
          format of the input file is as follows

\hspace{0.5cm} \textit{ki qi}

where \textit{ki} and \textit{qi} are, respectively, the degree at
layer 1 and the degree at layer 2 of node \textit{i}.

          If we consider two layers of a multiplex, and we denote by
          $k$ the degree of a node on the first layer and by $q$ the
          degree of the same node on the second layers, the
          inter-layer degree correlation function is defined as

          \begin{equation*}
            \overline{k}(q) = \frac{1}{N_{k}}\sum_{k'} k' P(k'|q)
          \end{equation*}
           
           where $P(k'|q)$ is the probability that a node with degree
           $q$ on the second layer has degree equal to $k'$ on the
           first layer, and $N_k$ is the number of nodes with degree
           $k$ on the first layer. The quantity $\overline{k}(q)$ is
           the expected degree at layer $1$ of node that have degree
           equal to $q$ on layer $2$. The dual quantity:

          \begin{equation*}
            \overline{q}(k) = \frac{1}{N_{q}}\sum_{q'} q' P(q'|k)
          \end{equation*}
           
           is the average degree on layer $2$ of nodes having degree
           $k$ on layer $1$.
}


\myreturn{The program prints on  \texttt{stdout} a list of lines in
           the format:

           \hspace{0.5cm} \textit{k $\overline{q}(k)$}           

           where \textit{k} is the degree on layer $1$ and
           $\overline{q}(k)$ is the average degree on layer $2$ of
           nodes having degree equal to $k$ on layer $1$. 

           The program also prints on \texttt{stderr} a list of lines in
           the format:

           \hspace{0.5cm} \textit{q $\overline{k}(q)$}           

           where \textit{q} is the degree on layer $2$ and
           $\overline{k}(q)$ is the average degree on layer $1$ of
           nodes having degree equal to $q$ on layer $2$. 
           }

\myreference{\refcorrelations

  \refgrowth

  \refnonlinear
  }

%% \input{./latex/structure/correlations/knn_q_from_degrees_log.tex}
\myprogram{{fit\_knn}}
          {power-law fit of the inter-layer degree correlation
          function.}  
          {$<$filein$>$ $<$alpha$>$}

\mydescription{Perform a power-law fit of the inter-layer degree
          correlation function:

          \begin{equation*}
           \overline{q}(k) = \frac{1}{N_{q}}\sum_{q'} q' P(q'|k)
          \end{equation*}

          where $k$ is the degree of a node on layer $1$, $q$ is the
          degree on layer $2$ and $P(q|k)$ is the probability that a
          node with degree $k$ on layer $1$ has degree $q$ on layer
          $2$. The program assumes that $\overline{q}(k)$ can be
          written in the form $a k^{b}$, and computes the two
          parameters $a$ and $b$ through a linear fit of the log-log
          plot of $\overline{q}(k)$.

          The input file \textit{filein} contains a list of lines in
          the format:

          \hspace{0.5cm} \textit{ki qi}

          where \textit{ki} is the degree of node $i$ at layer $1$
          and \textit{qi} is the degree of node $i$ at layer $2$. 

          The second parameter \textit{alpha} is the ratio of the
          progression used to generate the exponentially-distributed
          bins for the log-log plot. Typical values of \textit{alpha}
          are between $1.1$ and $2.0$.
          
          
          N.B.: The exponent $b$ computed with this method is known to
          be inaccurate.
}



\myreturn{The program prints on \texttt{stdout} the values of the
          parameters $a$ and $b$ of the power-law fit $\overline{q}(k)
          = a k^{b}$.}

\myreference{\refcorrelations

  \refgrowth

  \refnonlinear
  }


\chapter{Models of multi-layer networks}

\section{Null models}

\subsection{Null-models of node and layer activity}

\myprogram{{model\_hypergeometric.py}}
          {Hypergeometric node activity null model.} 
          {$<$layer\_N\_file$>$ $<$N$>$}

\mydescription{This is the hypergeometric model of node activation. In
          this model each layer has exactly the same number of active
          node of a reference multiplex network, but nodes on each
          layer are activated uniformly at random, thus destroying all
          inter-layer activity correlation patterns.

          The file \textit{layer\_N\_file} reports on the n-th line
          the number of active nodes on the n-th layer (starting from
          zero). The second parameter \textit{N} is the total number
          of active nodes in the multiplex.
 }


\myreturn{The program prints on \texttt{stdout} a node-layer list of lines in the
 format:

 \hspace{0.5cm} \textit{node\_i layer\_i}
 
 where \textit{node\_i} is the ID of a node and \textit{layre\_i} is
 the ID of a layer. This list indicates which nodes are active in
 which layer. For instance, the line:

 \hspace{0.5cm} \textit{24 3}

 indicates that the node with ID \textit{24} is active on
 layer \textit{3}.
}

\myreference{\refcorrelations}

\myprogram{{model\_MDM.py}}
          {Multi-activity Deterministic Model.} 
          {$<$Bi\_file$>$ $<$M$>$}

\mydescription{This is the Multi-activity Deterministic Model (MDM). 
           In this model each node $i$ is considered active if it was
           active in the reference multiplex, maintains the same value
           of node activity $B_i$ (i.e., the number of layers in which
           it was active) and is associated an activity vector sampled
           uniformly at random from the $M\choose{B_i}$ possible
           activity vectors with $B_i$ non-null entries.

          The file \textit{Bi\_file}  is in the format:
          
          \hspace{0.5cm} \textit{Bi N(Bi)}
          
          where \textit{Bi} is a value of node activity
          and \textit{N(Bi)} is the number of nodes which had node
          activity equaly to \textit{Bi} in the reference multiplex.
          
          The parameter \textit{M} is the number of layers in the
          multiplex.
}         


\myreturn{The program prints on \texttt{stdout} a distribution of
          bit-strings, in the format:

          \hspace{0.5cm} \textit{Bi bitstring count}

          where \textit{bitstring} is the activity
          bitstring, \textit{Bi} is the number of non-zero entries
          of \textit{bitstring} and \textit{count} is the number of
          times that \textit{bitstrings} appear in the null model.}


\myreference{\refcorrelations}

\myprogram{{model\_MSM.py}}
          {Multi-activity Stochastic Model.} 
          {$<$node\_Bi\_file$>$ $<$M$>$}

\mydescription{This is the Multi-activity Stochastic Model (MSM). 
           In this model each node $i$ is considered active if it was
           active in the reference multiplex, and is activated on
           each layer with a probability equal to $B_i/M$ where $B_i$
           was the activity of node $i$ in the reference multiplex.

          The file \textit{node\_Bi\_file}  is in the format:
          
          \hspace{0.5cm} \textit{node\_i Bi)}
          
          where \textit{Bi} is the value of node activity
          of \textit{node\_i} in the reference multiplex.
          
          
          The parameter \textit{M} is the number of layers in the
          multiplex.
}         

\myreturn{The program prints on \texttt{stdout} a node-layer list of lines in the
 format:

 \hspace{0.5cm} \textit{node\_i layer\_i}
 
 where \textit{node\_i} is the ID of a node and \textit{layre\_i} is
 the ID of a layer. This list indicates which nodes are active in
 which layer. For instance, the line:

 \hspace{0.5cm} \textit{24 3}

 indicates that the node with ID \textit{24} is active on
 layer \textit{3}.
}

\myreference{\refcorrelations}

\myprogram{{model\_layer\_growth.py}}
          {Layer growth with preferential activation model.} 
          {$<$layer\_N\_file$>$ $<$N$>$ $<$M0$>$ $<$A$>$ [RND]}

\mydescription{This is the model of layer growth with preferential
          node activation. In this model an entire new layer arrives
          at time $t$ and a number of nodes $N_t$ is activated ($N\_t$
          is equal to the number of nodes active on that layer in the
          reference multiplex). Then, each node $i$ of the new layer
          is activated with a probability:

          \begin{equation*}
          P_i(t) \propto A + B_i(t)
          \end{equation*}

          where $B_i(t)$ is the activity of node $i$ at time $t$
          (i.e., the number of layers in which node $i$ is active at
          time $t$) while $A>0$ is an intrinsic attractiveness.

          The file \textit{layer\_N\_file} reports on the n-th line
          the number of active nodes on the n-th layer.
          
          The parameter \textit{N} is the number of nodes in the
          multiplex, \textit{M0} is the number of layers in the
          initial network, \textit{A} is the value of
          node attractiveness.

          If the user specifies \texttt{RND} as the last parameter,
          the sequence of layers is }

\myreturn{The program prints on \texttt{stdout} a node-layer list of lines in the
 format:

 \hspace{0.5cm} \textit{node\_i layer\_i}
 
 where \textit{node\_i} is the ID of a node and \textit{layre\_i} is
 the ID of a layer. This list indicates which nodes are active in
 which layer. For instance, the line:

 \hspace{0.5cm} \textit{24 3}

 indicates that the node with ID \textit{24} is active on
 layer \textit{3}.
}

\myreference{\refcorrelations}



\section{Growing multiplex networks}

\subsection{Linear preferential attachment}

\myprogram{{nibilab\_linear\_delta}}
          {Multiplex linear preferential attachment model --
          Synchronous arrival.}  
          {$<$N$>$ $<$m$>$ $<$m0$>$ $<$outfile$>$ $<$a$>$ $<$b$>$ $<$c$>$ $<$d$>$}

\mydescription{Grow a two-layer multiplex network using the multiplex linear
          preferential attachment model by Nicosia, Bianconi, Latora,
          Barthelemy (NiBiLaB).

          The probability for a newly arrived node $i$ to create a
          link to node $j$ on layer $1$ is:
          
          \begin{equation*}
          \Pi_{i\to j}^{1} \propto ak\lay{1}_j + bk\lay{2}_j
          \end{equation*}

          and the dual probability for $i$ to create a link to $j$ on
          layer $2$ is:

          \begin{equation*}
          \Pi_{i\to j}^{2} \propto ck\lay{1}_j + dk\lay{2}_j
          \end{equation*}
          
          Each new node arrives at the same time on both layers.

          The (mandatory) parameters are as follows:

          \begin{itemize}

          \item \textbf{N} number of nodes in the final graph

          \item \textbf{m} number of new edges brought by each new node

          \item \textbf{m0} number of nodes in the initial seed
                    graph. \textit{m0} must be larger than of equal
                    to \textit{m}.

          \item \textbf{outfile} the name of the file which will contain the            
          
          \item \textbf{a,b,c,d} the coefficients of the attaching probability
          function

          \end{itemize}
 }


\myreturn{The program dumps on the file \texttt{outfile} the
 (undirected) edge list of the resulting network. Each line of the
 file is in the format:
  
  \hspace{0.5cm}\textit{src\_ID} \textit{dest\_ID}
  
  where \textit{src\_ID} and \textit{dest\_ID} are the IDs of the two
  endpoints of an edge.
}

\myreference{\refgrowth}

\myprogram{{nibilab\_linear\_delay}}
          {Multiplex linear preferential attachment model --
          Asynchronous arrival.}  
          {$<$N$>$ $<$m$>$ $<$m0$>$ $<$outfile$>$ $<$a$>$ $<$b$>$
          $<$c$>$ $<$d$>$ $<$beta$>$}

\mydescription{Grow a two-layer multiplex network using the multiplex linear
          preferential attachment model by Nicosia, Bianconi, Latora,
          Barthelemy (NiBiLaB).

          The probability for a newly arrived node $i$ to create a
          link to node $j$ on layer $1$ is:
          
          \begin{equation*}
          \Pi_{i\to j}^{1} \propto ak\lay{1}_j + bk\lay{2}_j
          \end{equation*}

          and the dual probability for $i$ to create a link to $j$ on
          layer $2$ is:

          \begin{equation*}
          \Pi_{i\to j}^{2} \propto ck\lay{1}_j + dk\lay{2}_j
          \end{equation*}
          
          Each new node arrives first on layer $1$, and its replica on
          the layer $2$ appears after a time delay $\tau$ sampled from
          the power-law function:

          \begin{equation*}
            P(\tau) \sim \tau^{-\beta}
          \end{equation*}

          The (mandatory) parameters are as follows:

          \begin{itemize}

          \item \textbf{N} number of nodes in the final graph

          \item \textbf{m} number of new edges brought by each new node

          \item \textbf{m0} number of nodes in the initial seed
                    graph. \textit{m0} must be larger than of equal
                    to \textit{m}.

          \item \textbf{outfile} the name of the file which will contain the            
          
          \item \textbf{a,b,c,d} the coefficients of the attaching probability
          function

          \item \textbf{beta} the exponent of the power-law delay
          function which determines the arrival of replicas on layer $2$

          \end{itemize}
 }


\myreturn{The program dumps on the file \texttt{outfile} the
 (undirected) edge list of the resulting network. Each line of the
 file is in the format:
  
  \hspace{0.5cm}\textit{src\_ID} \textit{dest\_ID}
  
  where \textit{src\_ID} and \textit{dest\_ID} are the IDs of the two
  endpoints of an edge.
}

\myreference{\refgrowth}

\myprogram{{nibilab\_linear\_delay\_mix}}
          {Multiplex linear preferential attachment model --
          Asynchronous arrival and randomly selected first layer.}  
          {$<$N$>$ $<$m$>$ $<$m0$>$ $<$outfile$>$ $<$a$>$ $<$b$>$
          $<$c$>$ $<$d$>$ $<$beta$>$}

\mydescription{Grow a two-layer multiplex network using the multiplex linear
          preferential attachment model by Nicosia, Bianconi, Latora,
          Barthelemy (NiBiLaB).

          The probability for a newly arrived node $i$ to create a
          link to node $j$ on layer $1$ is:
          
          \begin{equation*}
          \Pi_{i\to j}^{1} \propto ak\lay{1}_j + bk\lay{2}_j
          \end{equation*}

          and the dual probability for $i$ to create a link to $j$ on
          layer $2$ is:

          \begin{equation*}
          \Pi_{i\to j}^{2} \propto ck\lay{1}_j + dk\lay{2}_j
          \end{equation*}
          
          Each new node arrives on one of the two layers, chosen
          uniformly at random, and its replica on the other layer
          appears after a time delay $\tau$ sampled from the power-law
          function:

          \begin{equation*}
            P(\tau) \sim \tau^{-\beta}
          \end{equation*}

          The (mandatory) parameters are as follows:

          \begin{itemize}

          \item \textbf{N} number of nodes in the final graph

          \item \textbf{m} number of new edges brought by each new node

          \item \textbf{m0} number of nodes in the initial seed
                    graph. \textit{m0} must be larger than of equal
                    to \textit{m}.

          \item \textbf{outfile} the name of the file which will contain the            
          
          \item \textbf{a,b,c,d} the coefficients of the attaching probability
          function

          \item \textbf{beta} the exponent of the power-law delay
          function which determines the arrival of replicas on layer $2$

          \end{itemize}
 }


\myreturn{The program dumps on the file \texttt{outfile} the
 (undirected) edge list of the resulting network. Each line of the
 file is in the format:
  
  \hspace{0.5cm}\textit{src\_ID} \textit{dest\_ID}
  
  where \textit{src\_ID} and \textit{dest\_ID} are the IDs of the two
  endpoints of an edge.
}

\myreference{\refgrowth}

\myprogram{{nibilab\_linear\_random\_times}}
          {Multiplex linear preferential attachment model --
          Asynchronous arrival with randomly sampled arrival times on
          layer 2.}  
          {$<$N$>$ $<$m$>$ $<$m0$>$ $<$outfile$>$ $<$a$>$ $<$b$>$
          $<$c$>$ $<$d$>$ }

\mydescription{Grow a two-layer multiplex network using the multiplex linear
          preferential attachment model by Nicosia, Bianconi, Latora,
          Barthelemy (NiBiLaB).

          The probability for a newly arrived node $i$ to create a
          link to node $j$ on layer $1$ is:
          
          \begin{equation*}
          \Pi_{i\to j}^{1} \propto ak\lay{1}_j + bk\lay{2}_j
          \end{equation*}

          and the dual probability for $i$ to create a link to $j$ on
          layer $2$ is:

          \begin{equation*}
          \Pi_{i\to j}^{2} \propto ck\lay{1}_j + dk\lay{2}_j
          \end{equation*}
          
          Each new node arrives on layer $1$, but its replica on the
          other layer appears at a uniformly chosen random time in
          $[m0+1; N]$.


          The (mandatory) parameters are as follows:

          \begin{itemize}

          \item \textbf{N} number of nodes in the final graph

          \item \textbf{m} number of new edges brought by each new node

          \item \textbf{m0} number of nodes in the initial seed
                    graph. \textit{m0} must be larger than of equal
                    to \textit{m}.

          \item \textbf{outfile} the name of the file which will contain the            
          
          \item \textbf{a,b,c,d} the coefficients of the attaching probability
          function


          \end{itemize}
 }


\myreturn{The program dumps on the file \texttt{outfile} the
 (undirected) edge list of the resulting network. Each line of the
 file is in the format:
  
  \hspace{0.5cm}\textit{src\_ID} \textit{dest\_ID}
  
  where \textit{src\_ID} and \textit{dest\_ID} are the IDs of the two
  endpoints of an edge.
}

\myreference{\refgrowth}

%%\input{./latex/models/growth/nibilab_semilinear.tex}

\subsection{Non-linear preferential attachment}

\myprogram{{nibilab\_nonlinear}}
          {Multiplex non-linear preferential attachment model --
          Synchronous arrival.}  
          {$<$N$>$ $<$m$>$ $<$m0$>$ $<$outfile$>$ $<$alpha$>$ $<$beta$>$}

\mydescription{Grow a two-layer multiplex network using the multiplex non-linear
          preferential attachment model by Nicosia, Bianconi, Latora,
          Barthelemy (NiBiLaB).

          The probability for a newly arrived node $i$ to create a
          link to node $j$ on layer $1$ is:
          
          \begin{equation*}
          \Pi_{i\to j}^{1} \propto \frac{\left(k\lay{1}_j\right)^{\alpha}} 
          {\left(k\lay{2}_j\right)^{\beta}}
          \end{equation*}

          and the dual probability for $i$ to create a link to $j$ on
          layer $2$ is:

          \begin{equation*}
          \Pi_{i\to j}^{2} \propto \frac{\left(k\lay{2}_j\right)^{\alpha}} 
          {\left(k\lay{1}_j\right)^{\beta}}
          \end{equation*}

          Each node arrives simultaneously on both layers.
          

          The (mandatory) parameters are as follows:

          \begin{itemize}

          \item \textbf{N} number of nodes in the final graph

          \item \textbf{m} number of new edges brought by each new node

          \item \textbf{m0} number of nodes in the initial seed
                    graph. \textit{m0} must be larger than of equal
                    to \textit{m}.

          \item \textbf{outfile} the name of the file which will contain the            
          
          \item \textbf{alpha, beta} exponents of of the attaching probability
          function

          \end{itemize}
 }


\myreturn{The program dumps on the file \texttt{outfile} the
 (undirected) edge list of the resulting network. Each line of the
 file is in the format:
  
  \hspace{0.5cm}\textit{src\_ID} \textit{dest\_ID}
  
  where \textit{src\_ID} and \textit{dest\_ID} are the IDs of the two
  endpoints of an edge.
}

\myreference{\refgrowth}



\subsection{Utilities}

\myprogram{{node\_deg\_over\_time.py}}
          {Time evolution of the degree of a node in a growing graph.}  
          {$<$layer$>$ $<$arrival\_times$>$ $<$node\_id$>$
          [$<$node\_id$>$ ...]}

\mydescription{Compute the degree $k_{i}(t)$ of node $i$ in a growing
          network as a function of time. The file \textit{layer}
          contains the edge list of the final network. Each line of
          the file is in the format:
  
  \hspace{0.5cm}\textit{src\_ID} \textit{dest\_ID}
  
  where \textit{src\_ID} and \textit{dest\_ID} are the IDs of the two
  endpoints of an edge.
  
  The file \textit{arrival\_times} is a list of node arrival times, in
 the format:

 \hspace{0.5cm} \textit{time\_i node\_i}
 
 where \textit{time\_i} is the time at which \textit{node\_i} arrived
 in the graph. Notice that \textit{time\_i} must be an integer in the
 range [0, N-1], where N is the total number of nodes in the final
 graph.

 The third parameter \textit{node\_id} is the ID of the node whose
 degree over time will be printed on output. If more than
 one \textit{node\_id} is provided, the degrees over time of all the
 corresponding nodes are printed on output.
 }


\myreturn{The program prints on \texttt{stdout} a list of lines in the
 format:

 \hspace{0.5cm} \textit{t kit}
 
 where \textit{kit} is the degree of node \textit{i} at
 time \textit{t}. The first line of output is in the format:

 \hspace{0.5cm} \textit{\#\#\#\# node\_id}

 where \textit{node\_id} is the ID of node \textit{i}.

 If more than one \textit{node\_id}s is provided as input, the program
 prints the degree over time of all of them, sequentially.
 
}

\myreference{\refgrowth}


\section{Multiplex networks with inter-layer degree correlations}

\subsection{Models based on simulated annealing}

\myprogram{{tune\_rho}}
          {Construct a multiplex with prescribed inter-layer correlations.} 
          {$<$rank1$>$  $<$rank2$>$ $<$rho$>$ $<$eps$>$ $<$beta$>$ [RND|NAT|INV]}

\mydescription{This programs tunes the inter-layer degree correlation
          coefficient $\rho$ (Spearman's rank correlation) of two
          layers, by adjusting the inter-layer pairing of nodes. The
          files \textit{rank1} and \textit{rank2} are the rankings of
          nodes in the first and second layer, where the n-th line of
          the file contains the rank of the n-th node (the highest
          ranked node has rank equal to 1).

          The parameter \textit{rho} is the desired value of the
          Spearman's rank correlation coefficient, while \textit{eps}
          is the accuracy of \textit{rho}. For instance,
          if \textit{rho} is set equal to -0.25 and \textit{eps} is
          equal to 0.0001, the program stops when the configuration of
          node pairing corresponds to a value of $\rho$ which differs
          from -0.25 by less than 0.0001.

          The parameter \textit{beta} is the typical inverse
          temperature of simulated annealing. 

          If no other parameter is specified, or if the last parameter
          is \texttt{RND}, the program starts from a random pairing of
          nodes. If the last parameter is \texttt{NAT} then the
          program assumes that the initial pairing is the natural one,
          where the nodes have the same ID on both layers. Finally,
          if \texttt{INV} is specified, the initial pairing is the
          inverse pairing, i.e. the one where node 0 on layer 1 is
          paired with node N-1 on layer 2, and so on.

 }


\myreturn{The program prints on \texttt{stdout} a pairing, i.e. a list
of lines in the format:

\hspace{0.5cm} \textit{IDL1 IDL2}

where \textit{IDL1} is the ID of the node on layer 1 and \textit{IDL2}
is the corresponding ID of the same node on layer 2.
}

\myreference{\refcorrelations}

\myprogram{{tune\_qnn\_adaptive}}
          {Construct a multiplex with prescribed inter-layer correlations.} 
          {$<$degs1$>$  $<$degs2$>$ $<$mu$>$ $<$eps$>$ $<$beta$>$ [RND|NAT|INV]}

\mydescription{This programs tunes the inter-layer degree correlation
          exponent $\mu$. If we consider two layers of a multiplex,
          and we denote by $k$ the degree of a node on the first layer
          and by $q$ the degree of the same node on the second layers,
          the inter-layer degree correlation function is defined as:
          
          \begin{equation*}
            \overline{q}(k) = \sum_{q'} q' P(q'|k)
          \end{equation*}

           where $\overline{q}(k)$ is the average degree on layer $2$
           of nodes having degree $k$ on layer $1$. 

           The program assumes that we want to set the degree
           correlation function such that:

           \begin{equation*}
           \overline{q}(k) = a k^{\mu}
           \end{equation*}
           
           where the exponent of the power-law function is given by
           the user (it is indeed the parameter \textit{mu}), and
           successively adjusts the pairing between nodes at the two
           layers in order to obtain a correlation function as close
           as possible to the desired one. The files \textit{degs1}
           and \textit{degs2} contain, respectively, the degrees of
           the nodes on the first layer and on the second layer. 
           
           The parameter \textit{eps} is the accuracy of \textit{mu}.
           For instance, if \textit{mu} is set equal to -0.25
           and \textit{eps} is equal to 0.0001, the program stops when
           the configuration of node pairing corresponds to a value of
           the exponent $\mu$ which differs from -0.25 by less than
           0.0001.

          The parameter \textit{beta} is the typical inverse
          temperature of simulated annealing.

          If no other parameter is specified, or if the last parameter
          is \texttt{RND}, the program starts from a random pairing of
          nodes. If the last parameter is \texttt{NAT} then the
          program assumes that the initial pairing is the natural one,
          where the nodes have the same ID on both layers. Finally,
          if \texttt{INV} is specified, the initial pairing is the
          inverse pairing, i.e. the one where node 0 on layer 1 is
          paired with node N-1 on layer 2, and so on.

 }


\myreturn{The program prints on \texttt{stdout} a pairing, i.e. a list
of lines in the format:

\hspace{0.5cm} \textit{IDL1 IDL2}

where \textit{IDL1} is the ID of the node on layer 1 and \textit{IDL2}
is the corresponding ID of the same node on layer 2.
}

\myreference{\refcorrelations}



\chapter{Dynamics on multi-layer networks}

\section{Interacting opinions - Multilayer ising model}

%%%
%%% Layer activity
%%%

\myprogram{{multiplex\_ising}}
          {compute the coupled ising model in a multiplex with $2$ layers.}
          {$<$layer1$>$ $<$layer2$>$ $<$T$>$ $<$J$>$ $<\gamma>$ $<h^{[1]}>$ $<h^{[2]}>$ $<p_1>$ $<p_2>$ $<num epochs>$}

\mydescription{Compute and print the output of the ising dynamics on two coupled layers of a multiplex network.
  Files \textit{layer1}, \textit{layer2}, contain the (undirected) edge list of the two layer, and each
  line is in the format:
  
  \hspace{0.5cm}\textit{src\_ID} \textit{dest\_ID}
  
  where \textit{src\_ID} and \textit{dest\_ID} are the IDs of the two
  endpoints of an edge.

 $T$ is the value of thermal noise in the system, $J$ the value of peer pressure, $\gamma$ the relative ratio between internal coupling and peer pressure, $h^{[1]}$ and $h^{[2]}$ the external fields acting on the two layers, $p_1$ the probability for a spin on layer $1$ at $t=0$ to be up, $p_2$ the same probability for spins on layer $2$, $num epochs$ the number of epochs for the simulation.} 

\myreturn{One line, reporting all controlling parameter, the value of consensus in layer $1$ $m^{[1]}$, the value of consensus in layer $2$ $m^{[2]}$ and the coherence $C$.}

\myreference{\refising}


\section{Biased random walks}

\subsection{Stationary distribution}

%%%
%%% Layer activity
%%%

\myprogram{{statdistr2}}
          {compute the stationary distribution of additive, multiplicative and intensive biased walks in a multiplex with $2$ layers.}
          {$<$layer1$>$ $<$layer2$>$ $<overlapping network>$ $<$N$>$ $b_1$ $b_2$}

\mydescription{Compute and print the stationary distribution of additive, multiplicative and intensive biased walks in a multiplex with $2$ layers.
  Files \textit{layer1}, \textit{layer2}, contain the (undirected) edge list of the two layer, and each
  line is in the format:
  
  \hspace{0.5cm}\textit{src\_ID} \textit{dest\_ID}
  
  where \textit{src\_ID} and \textit{dest\_ID} are the IDs of the two
  endpoints of an edge.

  The file \textit{overlapping network} has also a third column indicating the number of times two nodes are connected across all layers.

 $N$ is the number of nodes, $b_1$ is the first bias exponent (the bias exponent for layer $1$ for additive and multiplicative walks, the bias exponent on the participation coefficient for intensive walks), $b_2$ is the second bias exponent (the bias exponent for layer $1$ for additive and multiplicative walks, the bias exponent on the participation coefficient for intensive walks).} 

\myreturn{N lines. In the n-th line we report the node ID, the stationary distribution of that node for additive walks with exponents $b_1$ and $b_2$, the stationary distribution for multiplicative walks with exponents $b_1$ and $b_2$, the stationary distribution for multiplicative walks with exponents $b_1$ and $b_2$, the values of the bias exponents $b_1$ and $b_2$.}

\myreference{\refbiased}


\subsection{Entropy rate}

%%%
%%% Layer activity
%%%

\myprogram{{entropyrate2add}}
          {compute the entropy rate of additive biased walks in a multiplex with $2$ layers.}
          {$<$layer1$>$ $<$layer2$>$ $<overlapping network>$ $<$N$>$ $b_1$ $b_2$}

\mydescription{Compute and print the entropy rate of an additive biased walk in a multiplex with $2$ layers and bias parameters $b_1$ and $b_2$.
  Files \textit{layer1}, \textit{layer2}, contain the (undirected) edge list of the two layer, and each
  line is in the format:
  
  \hspace{0.5cm}\textit{src\_ID} \textit{dest\_ID}
  
  where \textit{src\_ID} and \textit{dest\_ID} are the IDs of the two
  endpoints of an edge.

  The file \textit{overlapping network} has also a third column indicating the number of times two nodes are connected across all layers.

 $N$ is the number of nodes, $b_1$ is the degree-biased exponent for layer $1$, $b_2$ is the degree-biased exponent for layer $2$.} 

\myreturn{One line, reporting the value of the entropy rate $h$ of an additive biased random walks with $b_1$ and $b_2$ as bias exponents, $b_1$ and $b_2$.}

\myreference{\refbiased}

%%%
%%% Layer activity
%%%

\myprogram{{entropyrate2mult}}
          {compute the entropy rate of multiplicative biased walks in a multiplex with $2$ layers.}
          {$<$layer1$>$ $<$layer2$>$ $<overlapping network>$ $<$N$>$ $b_1$ $b_2$}

\mydescription{Compute and print the entropy rate of a multiplicative biased walk in a multiplex with $2$ layers and bias parameters $b_1$ and $b_2$.
  Files \textit{layer1}, \textit{layer2}, contain the (undirected) edge list of the two layer, and each
  line is in the format:
  
  \hspace{0.5cm}\textit{src\_ID} \textit{dest\_ID}
  
  where \textit{src\_ID} and \textit{dest\_ID} are the IDs of the two
  endpoints of an edge.

  The file \textit{overlapping network} has also a third column indicating the number of times two nodes are connected across all layers.

 $N$ is the number of nodes, $b_1$ is the degree-biased exponent for layer $1$, $b_2$ is the degree-biased exponent for layer $2$.} 

\myreturn{One line, reporting the value of the entropy rate $h$ of an multiplicative biased random walks with $b_1$ and $b_2$ as bias exponents, $b_1$ and $b_2$.}

\myreference{\refbiased}

%%%
%%% Layer activity
%%%

\myprogram{{entropyrate2int}}
          {compute the entropy rate of intensive biased walks in a multiplex with $2$ layers.}
          {$<$layer1$>$ $<$layer2$>$ $<overlapping network>$ $<$N$>$ $b_1$ $b_2$}

\mydescription{Compute and print the entropy rate of an intensive biased walks in a multiplex with $2$ layers and bias parameters $b_p$ and $b_o$.
  Files \textit{layer1}, \textit{layer2}, contain the (undirected) edge list of the two layer, and each
  line is in the format:
  
  \hspace{0.5cm}\textit{src\_ID} \textit{dest\_ID}
  
  where \textit{src\_ID} and \textit{dest\_ID} are the IDs of the two
  endpoints of an edge.

  The file \textit{overlapping network} has also a third column indicating the number of times two nodes are connected across all layers.

 $N$ is the number of nodes, $b_p$ is the biased exponent on the participation coefficient, $b_o$ is the biased exponent on the overlapping degree.} 

\myreturn{One line, reporting the value of the entropy rate $h$ of an intensive biased random walks with $b_p$ and $b_o$ as bias exponents, $b_p$ and $b_o$.}

\myreference{\refbiased}


\end{document}
